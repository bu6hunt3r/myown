\usepackage[utf8]{inputenc}

\usepackage{graphicx}
\usepackage{chronology}
\usepackage{tikz}

\definecolor{sLightBlue}{RGB}{52,101,164}
\definecolor{sBlue}{RGB}{24,80,149}
\definecolor{sDarkBlue}{RGB}{8,60,127}
\definecolor{sOrange}{RGB}{224, 135, 21}
\definecolor{sDarkGray}{RGB}{50,50,50}
\definecolor{sDarkerGray}{RGB}{28,28,28}
\definecolor{sLightGray}{RGB}{178,178,178}
\definecolor{sSyntaxNDKeyword}{RGB}{255,100,255}
\definecolor{sSyntaxString}{RGB}{100,255,100}
\definecolor{sSyntaxComment}{RGB}{255,255,128}
\definecolor{sSyntaxKeyword}{RGB}{72,210,255}
\definecolor{sNSyntaxKeyword}{RGB}{255,125,255}

\definecolor{pbblue}{HTML}{0A75A8} % color for progress bar and circle
\makeatletter
\def\progressbar@progressbar{}      % the progressbar
\newcount\progressbar@tmpcounta     % auxiliary counter a
\newcount\progressbar@tmpcountb     % auxiliary counter b
\newdimen\progressbar@pbht          % progressbar height
\newdimen\progressbar@pbwd          % progressbar width
\newdimen\progressbar@rcircle       % radius for circle
\newdimen\progressbar@tmpdim        % auxiliary dimen

\progressbar@pbwd=\linewidth
\progressbar@pbht=1pt
\progressbar@rcircle=2.5pt

% the progressbar
\def\progressbar@progressbar{%

    \progressbar@tmpcounta=\insertframenumber
    \progressbar@tmpcountb=\inserttotalframenumber
    \progressbar@tmpdim=\progressbar@pbwd
    \multiply\progressbar@tmpdim by \progressbar@tmpcounta
    \divide\progressbar@tmpdim by \progressbar@tmpcountb

    \begin{tikzpicture}
        \draw[pbblue!30,line, width=\progressbar@pbht]
        (0pt, 0pt) -- ++ (\progressbar@pbwd,0pt);

        \filldraw[pbblue!30] %
            (\the\dimexpr\progressbar@tmpdim-\progressbar@rcircle\relax, .5\progressbar@pbht) circle (\progressbar@rcircle);
    
        \node[draw=pbblue!30,text width=3.5em,align=center,inner sep=1pt, text=pbblue!70,anchor=east] at (0,0) {\insertframenumber/\inserttotalframenumber};
    \end{tikzpicture}%
 }
\makeatother

\newcommand{\background}[4]{%
    \begin{pgfonlayer}{background}
        % Left-top-corner of background rectangle
        \path (#1.west |- #2.north) node (a1) {};
        % Right-bottom corner of the background rectanle
        
        %\path (#4.west |- #3.south)+(-0.5,-1) node (a2) {};
        
        \path (#3.south |- #4.east) node (a2) {};
        % Draw the background
        \path[fill=gray!80,rounded corners, draw=black!50, dashed] ([xshift=-1.0em,yshift=1.0em]a1) rectangle ([xshift=3.0em,yshift=-3.0em]a2);
    \end{pgfonlayer}
}

\tikzset{% 
register/.style={rectangle,draw=white,fill=black},
    byte/.style={rectangle,draw=white,fill=black},
    header/.style={rectangle,draw=white,fill=black},
    important/.style={fill=sOrange},
    highlighted/.style={fill=sLightBlue},
    highlighted3/.style={fill=sDarkBlue},
    highlighted2/.style={fill=sBlue},
    plevels/.style={rectangle,draw=white,fill=black}, 
    plevellist/.style={
        text depth=0.5ex, 
        text height=1.5ex, 
        minimum height=1.5, 
        nodes in empty cells, 
        matrix of nodes,
        row sep=-\pgflinewidth,
        column sep=-\pgflinewidth,
        nodes={plevels, align=left}, 
        row 1/.style={nodes={highlighted, align=center}},
        column 1/.style={font=\ttfamily},
        column 2/.style={font=\ttfamily}
    },
    registerlist2/.style={
        text depth=0.5ex, 
        text height=1.5ex, 
        minimum height=1.5, 
        nodes in empty cells, 
        matrix of nodes,
        row sep=-\pgflinewidth,
        column sep=-\pgflinewidth,
        nodes={register, align=left}, 
        row 1/.style={nodes={highlighted, align=center}}, 
        column 1/.style={text width=3.5em, font=\ttfamily},
        column 2/.style={text width=17em, font=\ttfamily},
        column 3/.style={text width=3.5em, font=\ttfamily}
    },
    conditionlist/.style={
        text depth=0.5ex, 
        text height=1.5ex, 
        minimum height=1.5, 
        matrix of nodes,
        row sep=-\pgflinewidth,
        column sep=-\pgflinewidth,
        nodes={register, align=left}, 
        row 1/.style={nodes={highlighted, align=center}}, 
        column 1/.style={text width=3em, font=\ttfamily},
        column 2/.style={text width=10em, font=\ttfamily},
        column 3/.style={text width=7.5em, font=\ttfamily}
    },
    registerlist/.style={
        minimum width=5em,
        matrix of nodes,
        row sep=-\pgflinewidth,
        column 1/.style={font=\ttfamily}
    },
    headerlist/.style={
        minimum width=12em, 
        minimum height=2em,
        matrix of nodes,
        row sep=-\pgflinewidth,
        column 1/.style={font=\ttfamily}
    },
    bytes/.style={
        minimum width=2em,
        matrix of nodes,
        nodes={byte},
        row sep=-\pgflinewidth,
        column sep=-\pgflinewidth,
        font=\ttfamily
    },
    bytes2/.style={
        minimum width=1.0em,
        minimum height=1.0em,
        matrix of nodes,
        row sep=-\pgflinewidth,
        column sep=-\pgflinewidth,
        font=\ttfamily 
    }
}
\usetikzlibrary{shapes, arrows.meta, decorations.pathreplacing,angles,quotes,matrix,positioning, arrows, calc, shadows}
\setbeamertemplate{background}
{\includegraphics[width=\paperwidth,height=\paperheight,keepaspectratio]{resources/logos.png}} 

\setbeamercolor{background canvas}{bg=sDarkerGray, fg=white}
\setbeamercolor{frametitle}{bg=sDarkerGray, fg=white}
\setbeamercolor{alerted text}{ fg=sOrange , bg=sDarkerGray }

\setbeamercolor{titlelike}{fg=white}
\setbeamercolor{author}{fg=white}
\setbeamercolor{date}{fg=white}
\setbeamercolor{institute}{fg=white}
\setbeamercolor{structure}{fg=white}
\setbeamercolor{progress bar}{fg=sLightBlue}
\setbeamercolor{progress bar in section page}{bg=sLightGray}
\setbeamercolor{normal text}{fg=white}
\setbeamertemplate{itemize items}{\textbullet}

\setmonofont{Source Code Pro}

\usepackage{listings}
\lstset{
           basicstyle=\linespread{1.1}\color{white}\scriptsize\ttfamily,
           keywordstyle=\color{sSyntaxKeyword}\bfseries,
           ndkeywordstyle=\color{sNSyntaxKeyword}\bfseries,
           stringstyle=\color{sSyntaxString},
           commentstyle=\color{sSyntaxComment},
           %identifierstyle=\color{sSyntaxIdentifier},
            %showstringspaces=false,
                                        % Dies ist ein Python-Skript
            %tabsize=4,                          % etwas kleinere Tabs
            %extendedchars=true,                 % Sonderzeichen erlauben
            % Zeilennummern
            %numberstyle=\scriptsize,            % kleinere Schriftart
	    frame=single,                       % Rahmen
            %frame=lr,
            backgroundcolor=\color{black}, % Hintergrundfarbe
            breaklines=true,                    % breche "lange Zeilen" um...
            captionpos=b,
            %xrightmargin=-29pt,
            %xleftmargin=-29pt,
           %  framexleftmargin=60pt,
           %  framexrightmargin=30pt,
             numberstyle=\tiny,
        %framextopmargin=3pt,
        %framexbottommargin=3pt, 
        %rulesep=30pt,
        %frame=tlbr, 
        %framerule=0pt,
        %framesep=0pt
        }
        \lstdefinestyle{clist}{
            basicstyle=\scriptsize\fontserif,
        }
        \lstdefinelanguage{bash}{
        keywords={readelf, ropper, cat, hexdump, as, ld, objcopy},
        ndkeywords={asm, arch, file, No},
        %backgroundcolor=\color{bash_bg},
          comment=[l]{\#},
          morestring=[b]',
          morestring=[b]"
        }
        \lstdefinelanguage{JavaScript}{
          keywords={typeof, new, true, false, catch, function, return, null, catch, switch, var, if, in, while, do, else, case, break},
          ndkeywords={class, export, boolean, throw, implements, import, this},
          sensitive=false,
          comment=[l]{//},
          morecomment=[s]{/*}{*/},
          morestring=[b]',
          morestring=[b]"
        }
         \lstdefinelanguage{assembly}{
          keywords={strb, eor, eors,svc, swi, push, add, adc, addgt, addeq, sub, sbc, subs, suble, str, mov,movs,adr, ldr, ldmia, ldmib, str, stmdb, stmda, bl, b, blx, bne, pop, push, bx, it, ite, itte, ittee, gt, eq, movne},
          ndkeywords={r0,r1,r2,r3,r4,r5,r6,r7,r8,r9,r10,r11,r12,r13,r14,r15,lr, pc,sp,ip,fp},
          sensitive=false,
          comment=[l]{\#},
          comment=[l]{;},
          morecomment=[s]{/*}{*/},
          morestring=[b]',
          morestring=[b]"
        }
        \lstdefinelanguage{radare}{
          keywords={rax2, rabin2, rasm2, rahash2, r2pm, r2, radare2, rafind2},
          ndkeywords={i,s,zz,z,X,m,L,a,aa,aaaa,aaa,AA},
          sensitive=false,
          comment=[l]{\#},
          morestring=[b]',
          morestring=[b]"
        }
        \lstdefinestyle{intermezzo}{
            basicstyle=\small\fontsans,
            numbers=none,
            language=bash,
            backgroundcolor=\color{yellow},
        }
\usepackage{booktabs}
\usepackage{pgfplots}
\usepackage{color, colortbl}
\usepgfplotslibrary{dateplot}
\metroset{progressbar=head}

\title{Spectre Meltdown}
\date{\today}
\author{ Felix C. Koch }

